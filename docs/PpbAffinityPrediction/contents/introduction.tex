\section{Introduction}
The binding of two proteins can be viewed as a reversible and rapid process in an equilibrium that is
governed by the law of mass action. Binding affinity is the strength of the interaction between two
(or more than two) molecules that bind reversibly (interact). It is translated into physico-chemical
terms in the dissociation constant Kd, the latter being the concentration of free protein at which
half of all binding sites of the second protein type are occupied \cite{Kastritis}. \\
Predicting the affinity of protein–protein complexes has been a topic of active research for more
than two decades. The availability of experimental data on binding affinity prompted researchers to
explore the principles and develop methods for prediction \cite{Gromiha}. However, the amount of experimentally
observed three-dimensional protein-protein complexes with known binding constants still remains
extremely limited, which complicates the application of modern deep learning methods in this
task. The most recent computational research on PPI binding affinity prediction is mainly built
around the idea of utilizing standard statistical and machine learning methods
trained on various handcrafted descriptors such as QSAR features \cite{Yang}, inter-residue contacts and
buried surface area , surface tension area and hydrophobicity, and sequence-based descriptors. \\
Recent advances in adjacent computational problems such as protein structure prediction or
protein-protein interaction prediction demonstrated how powerful deep learning methods can
be as long as enough training data is provided and correct inductive biases are set up. \\
In this work, we will apply geometric deep learning methods for predicting protein-protein binding
affinity. We believe that geometry is a clue for understanding protein-protein interactions, and aim
to notably move forward the state of the art in binding affinity prediction with the aid of graph
neural networks. To the best of our knowledge, geometric deep learning methods have never been
applied to the protein-protein binding affinity prediction problem so far.
\subsection{Objectives}
Three main objectives of this work can be formulated as follows:
\begin{itemize}
\item Refine PDBbind \cite{database} data and a standard binding affinity dataset, and compile a novel
benchmark of PPIs with known binding affinity values
\item Employ graph-learning toolset to predict binding affinities of PPIs from the new dataset
\item Benchmark the resulting method against existing state-of-the-art approaches
\end{itemize}

