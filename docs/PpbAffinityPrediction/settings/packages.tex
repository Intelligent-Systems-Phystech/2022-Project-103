\usepackage[T2A]{fontenc}			% кодировка
\usepackage[utf8]{inputenc}
\usepackage[english, russian]{babel}	% локализация и переносы

\usepackage{graphicx}

\setlength\parindent{0pt}

\sloppy                                    % строго соблюдать границы текста
\linespread{1.3}                           % коэффициент межстрочного интервала
\setlength{\parskip}{0.5em}                % вертик. интервал между абзацами

\usepackage{lipsum}
\usepackage{easy-todo} % Добавление задач с помощью \todo{}
\usepackage{csquotes} % Для "правильных" кавычек

\usepackage[iso, german]{isodate} % Für eine deutsche Formatierung des Abgabedatums / Eidesstattlicher Erkärung
\usepackage[style=apa, backend=biber, sortlocale=de_RU]{biblatex} % Biber backend für Literaturverzeichnis
\addbibresource{literatur/bibliography.bib} % Интеграция литературы

\DeclareLanguageMapping{german}{german-apa} % Anpassen Spracheinstellungen im Literaturverzeichnis.
\usepackage[activate={true,nocompatibility},
	final,
	tracking=true,
	kerning=true,
	expansion=true,
	spacing=true,
	factor=1050,
	stretch=25,
	shrink=10]{microtype} % Für die Feineinstellung der Zeichensetzung.
\usepackage{booktabs}
\usepackage{appendix}
\usepackage[rflt]{floatflt}
\usepackage{fancyvrb}
% \usepackage[hidelinks]{hyperref} % Klickbare aber nicht markierte Links im PDF
\usepackage{setspace}
% \usepackage{fancyhdr} % Für schönere Kopf-/Fußzeilen und Fußnoten.
% \usepackage[right=4 cm, left=2.5 cm, top=2.5 cm, bottom=3 cm]{geometry} % Seitenränder
\usepackage{pbox}
\usepackage{tabulary}

\hypersetup{
    pdfstartview=FitH,
    colorlinks=true,
    linkcolor=blue,
    filecolor=magenta,      
    urlcolor=cyan,
}

\usepackage{titleps}                       % колонтитулы

\newpagestyle{main}{
  \setheadrule{.4pt}                      
  \sethead{\coursename}{}{\hyperlink{intro}{\;Назад к содержанию}}
  \setfootrule{.4pt}                       
  \setfoot{\coursedate \; ФПМИ МФТИ}{}{\thepage} 
}      

\pagestyle{main}

\sloppy                                    % строго соблюдать границы текста
\linespread{1.3}                           % коэффициент межстрочного интервала
\setlength{\parskip}{0.5em}                % вертик. интервал между абзацами

\setcounter{secnumdepth}{0}                % отключение нумерации разделов

\usepackage{listings}
\usepackage{color}

\title{Syntax Highlighting in LaTeX with the listings Package}
\author{writeLaTeX}

\definecolor{mygreen}{rgb}{0,0.6,0}
\definecolor{mygray}{rgb}{0.5,0.5,0.5}
\definecolor{mymauve}{rgb}{0.58,0,0.82}

\lstset{ %
  backgroundcolor=\color{white},   % choose the background color
  basicstyle=\footnotesize,        % size of fonts used for the code
  breaklines=true,                 % automatic line breaking only at whitespace
  captionpos=b,                    % sets the caption-position to bottom
  commentstyle=\color{mygreen},    % comment style
  escapeinside={\%*}{*)},          % if you want to add LaTeX within your code
  keywordstyle=\color{blue},       % keyword style
  stringstyle=\color{mymauve},     % string literal style
}