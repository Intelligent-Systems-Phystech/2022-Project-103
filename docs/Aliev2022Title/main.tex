\documentclass{article}
\usepackage{arxiv}

\usepackage[utf8]{inputenc}
\usepackage[english, russian]{babel}
\usepackage[T1]{fontenc}
\usepackage{url}
\usepackage{booktabs}
\usepackage{amsfonts}
\usepackage{nicefrac}
\usepackage{microtype}
\usepackage{lipsum}
\usepackage{graphicx}
\usepackage{natbib}
\usepackage{doi}



\title{A template for the \emph{arxiv} style}

\author{ Alen Aliev	\\
	MIPT\\
	Moscow, Russia \\
	\texttt{aliev.ae@phystech.edu} \\
	%% examples of more authors
	\And
	Ilia Igashov\\
	\And
	Arne Schneuing
	\\
	\\
	\texttt{} \\
	%% \AND
	%% Coauthor \\
	%% Affiliation \\
	%% Address \\
	%% \texttt{email} \\
	%% \And
	%% Coauthor \\
	%% Affiliation \\
	%% Address \\
	%% \texttt{email} \\
	%% \And
	%% Coauthor \\
	%% Affiliation \\
	%% Address \\
	%% \texttt{email} \\
}
\date{}

\renewcommand{\shorttitle}{\textit{arXiv} Template}

%%% Add PDF metadata to help others organize their library
%%% Once the PDF is generated, you can check the metadata with
%%% $ pdfinfo template.pdf
\hypersetup{
pdftitle={A template for the arxiv style},
pdfsubject={q-bio.NC, q-bio.QM},
pdfauthor={},
pdfkeywords={First keyword, Second keyword, More},
}

\begin{document}
\maketitle

% \begin{abstract}
Proteins are involved in several biological reactions by means of interactions with other proteins or
with other molecules such as nucleic acids, carbohydrates, and ligands. Among these interaction
types, protein–protein interactions (PPIs) are considered to be one of the key factors as they are
involved in most of the cellular processes. \\
In this work we aim to compile a novel benchmark of PPIs with known binding affinity values from refined data and benchmark the resulting deep learning geometry method against existing state-of-the-art approaches.
% \end{abstract}


\end{document}